\section{Motivation}

\hammerpage%

\frame{%
    \centering
    \begin{figure}
        \includegraphics[width=.8\textwidth]{img/gorilla.png}
        \caption{%
            via: BBC News (\url{%
                https://www.bbc.co.uk/news/technology-33347866
            })
        }
    \end{figure}
}

\frame{%
    \alert{Reliability and frailty}
    \begin{itemize}
        \item[] \fullcite{Torralba2011}
    \end{itemize}
}

% TODO
% \frame{%
%    Paradigm flip diagram
% }

\section{Generating artificial data}

\frame{%
    \centering
    \begin{figure}
        \includegraphics[width=.3\textwidth]{img/faces/0.jpeg}\hfill%
        \includegraphics[width=.3\textwidth]{img/faces/1.jpeg}\hfill%
        \includegraphics[width=.3\textwidth]{img/faces/2.jpeg}
        \caption{via: \url{https://thispersondoesnotexist.com}}
    \end{figure}
}

\frame{%
    \begin{figure}
        \includegraphics[width=\textwidth]{img/resume.png}
        \caption{via: \url{https://thisresumedoesnotexist.com}}
    \end{figure}
}

\frame{%
    \begin{figure}
        \includegraphics[width=.45\textwidth]{img/cats/0.jpeg}\hfill%
        \includegraphics[width=.45\textwidth]{img/cats/1.jpeg}
        \caption{via: \url{https://thiscatdoesnotexist.com}}
    \end{figure}
}

\frame{%
    \alert{Anscombe's quartet}\vfill
    \includegraphics[width=\textwidth]{img/anscombes.pdf}\vfill
}

\frame{%
    \alert{The Datasaurus dozen}\vfill
    \includegraphics[width=\textwidth]{img/datasaurus.pdf}\vfill
}

\frame{%
    \resizebox{\textwidth}{!}{%
        \begin{tikzpicture}

            \fill[gray!15] (-4, 5) rectangle (-2, 8.5);
            \fill[cyan!35] (-2, 5) rectangle (0, 8.5);
            \fill[magenta!5] (0, 5) rectangle (2, 8.5);
            \path (-2, 5) pic {fullcolumn=10}
                  (0, 5) pic {fullcolumn=10}
                  (2, 5) pic {fullcolumn=10};
            \node (dataset) at (-1, 4.5) {};

            \fill[pattern=north west lines, pattern color=cyan!35]
                (6, 0) rectangle (8, 3.15);
            \fill[pattern=north east lines, pattern color=magenta!50]
                (8, 0) rectangle (10, 3.15);
            \fill[pattern=north west lines, pattern color=gray!50]
                (10, 0) rectangle (12, 3.15);
            \path (8, 0) pic {fullcolumn=9}
                  (10, 0) pic {fullcolumn=9}
                  (12, 0) pic {fullcolumn=9};
            \node (similar) at (5.5, 2) {};

            \draw[->, orange, ultra thick]
                (dataset.south) to [out=290, in=180] (similar.west);
            \draw[->, orange, ultra thick]
                (7, 3.5) to [out=100, in=0] (2.5, 6)
                node[right=20pt, below=50pt]
                {\Large\color{orange} make `similar'};
        \end{tikzpicture}
    }
}

\frame{%
    \huge{%
        Given an algorithm, how can one find data for which it performs well?
    }
}

