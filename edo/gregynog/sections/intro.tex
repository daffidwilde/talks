\hammerpage%

\frame{%
    \centering
    \begin{figure}
        \includegraphics[width=.8\textwidth]{../img/gorilla.png}
        \caption{%
            via: BBC News (\url{%
                https://www.bbc.co.uk/news/technology-33347866
            })
        }
    \end{figure}
}

\frame{%
    \alert{Reliability}
    \begin{itemize}
        \item[] \fullcite{Hyndman2014}
    \end{itemize}

    \alert{Frailty}
    \begin{itemize}
        \item[] \fullcite{Torralba2011}
    \end{itemize}
}

\frame{%
    \centering
    \resizebox{!}{.9\paperheight}{%
        \begin{tikzpicture}

    \begin{pgfonlayer}{background}
    % Data
    \node[%
        ellipse,
        minimum height=3cm,
        minimum width=7cm,
        fill=cyan!15,
        label={below:\Large Data}
        ] at (0, 0) {};

    % Algorithms
    \node[%
        rectangle,
        rounded corners=0.25cm,
        minimum height=1cm,
        minimum width=6cm,
        fill=orange!15,
        label={\Large Algorithms}
    ] at (0, -6) {};
    \end{pgfonlayer}

    \pause
    \node[circle, fill=cyan, thick, inner sep=2pt, minimum size=2mm]
        (d) at (2, 0) {};

    \pause
    \node[circle] (q1) at (2, -2) {?};
    \draw[->, thick] (d.south) -- (q1.north);

    \pause
    \node[circle] (q2a) at (1, -3) {?};
    \node[circle] (q2b) at (3, -3) {?};
    \draw[->, thick] (q1.south west) -- (q2a.north east);
    \draw[->, thick] (q1.south east) -- (q2b.north west);

    \pause
    \node[circle] (q3) at (2, -4) {?};
    \draw[->, thick] (q2a.south east) -- (q3.north west);
    \draw[thick] (q2a.south west) -- ++(-0.5, -0.5);

    \pause
    \node[circle, fill=orange, thick, inner sep=2pt, minimum size=2mm]
        (a) at (2, -6) {};

    \draw[->, thick] (q3.south) -- (a.north);

    \pause
    \node[circle, fill=orange, thick, inner sep=2pt, minimum size=2mm]
        (a2) at (-2, -6) {};

    \pause
    \node[%
        ellipse,
        fill=cyan!50,
        minimum height=0.75cm,
        minimum width=1cm
    ] (datasets) at (-1.5, 0) {};
    \foreach \position in {(-1.7, 0), (-1.5, 0.1), (-1.3, -0.1)} {%
        \fill[cyan] \position circle (1mm);
    };

    \pause
    \node[%
        ellipse,
        fill=cyan!35,
        minimum height=1.3cm,
        minimum width=2.5cm,
    ] (data) at (-1, 0) {};

    \node[%
        ellipse,
        fill=cyan!50,
        minimum height=0.75cm,
        minimum width=1cm
    ] (datasets) at (-1.5, 0) {};
    \foreach \position in {(-1.7, 0), (-1.5, 0.1), (-1.3, -0.1)} {%
        \fill[cyan] \position circle (1mm);
    };

    \pause
    \draw[->, thick]
        (a2.north west) to [out=130, in=210] (data.south west);
\end{tikzpicture}

    }
}

\section{Generating artificial data}

\frame{%
    \centering
    \begin{figure}
        \includegraphics[width=.3\textwidth]{../img/faces/0.jpeg}\hfill%
        \includegraphics[width=.3\textwidth]{../img/faces/1.jpeg}\hfill%
        \includegraphics[width=.3\textwidth]{../img/faces/2.jpeg}
        \caption{via: \url{https://thispersondoesnotexist.com}}
    \end{figure}
}

\frame{%
    \alert{Anscombe's quartet}\vfill
    \includegraphics[width=\textwidth]{../img/anscombes.pdf}\vfill
}

\frame{%
    \resizebox{\textwidth}{!}{%
        \begin{tikzpicture}

    % Dataset
    \fill[gray!15] (-4, 5) rectangle (-2, 8.5);
    \fill[cyan!35] (-2, 5) rectangle (0, 8.5);
    \fill[magenta!5] (0, 5) rectangle (2, 8.5);
    \path (-2, 5) pic {fullcolumn=10}
          (0, 5) pic {fullcolumn=10}
          (2, 5) pic {fullcolumn=10};
    \node (dataset) at (-1, 4.5) {};

    % A similar dataset
    \fill[pattern=north west lines, pattern color=cyan!35]
        (6, 0) rectangle (8, 3.15);
    \fill[pattern=north east lines, pattern color=magenta!50]
        (8, 0) rectangle (10, 3.15);
    \fill[pattern=north west lines, pattern color=gray!50]
        (10, 0) rectangle (12, 3.15);
    \path (8, 0) pic {fullcolumn=9}
        (10, 0) pic {fullcolumn=9}
        (12, 0) pic {fullcolumn=9};
    \node (similar) at (5.5, 2) {};

    \draw[->, orange, ultra thick]
        (dataset.south) to [out=290, in=180] (similar.west);
    \draw[->, orange, ultra thick]
        (7, 3.5) to [out=100, in=0] (2.5, 6) node[right=20pt, below=50pt]
        {\Large\color{orange} make `similar'};

\end{tikzpicture}

    }
}

\frame{%
    \huge{%
        Given an algorithm, how can one find sets of data for which it performs
        well?
    }
}

